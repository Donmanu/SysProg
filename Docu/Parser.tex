% °_°_°_°_°_°_°_°_°_°_°_
%   PREAMBEL / IMPORTS
% °_°_°_°_°_°_°_°_°_°_°_
\documentclass[
a4paper,   %|letterpaper|a5paper|b5paper|legalpaper|executivepaper,
11pt,      %|10pt|12pt,
oneside,   %|twoside,
onecolumn, %|twocolumn,
final      %|draft,     wie Bilder eingebunden sind
]{article}

% Deutsche Sprache:
\usepackage[utf8]{inputenc} % Ermöglicht die direkte Eingabe der Umlaute.
\usepackage[T1]{fontenc}    % Wird u.a. für das Trennen von Wörtern mit Umlauten genutzt.
\usepackage[ngerman]{babel} % Wird benötigt um deutsche Bezeichnungen zu erhalten. Zum Beispiel 'Inhaltsverzeichnis'
                            % anstelle von 'Table of contents'.  Auch werden dann die Wörter gemäß der _neuen_
                            % Rechtschreibung getrennt.
\addto\captionsngerman{\renewcommand{\abstractname}{Abstract}} % Aber doch Abstract statt "Zusammenfassung"

% Bilder:
\usepackage{graphicx}    % Besserer Bilderimport
\usepackage{wrapfig}     % Erlaube Text um Bilder
\usepackage[font={small,it},labelfont={normal}]{caption}  % Bessere Bild-/Code-/Tabellenuntertitel

% Schrift und Optik:
\usepackage{geometry}
\usepackage{concrete}    % Ganz hübsche Schriftart von Donald himself
\geometry{a4paper,left=28mm,right=28mm, top=38mm, bottom=38mm}
\usepackage{setspace}
\onehalfspacing          % Anderhalbfacher Zeilenabstand
\widowpenalty = 10000
\clubpenalty = 10000

% Code:
\usepackage{color}       % Vor allem für farbigen Code
\definecolor{lightgray}{rgb}{ .95, .95, .95 }
\definecolor{shadow}{rgb}   { .75, .75, .75 }
\definecolor{darkgray}{rgb} { .4 , .4 , .4  }
\definecolor{blue}{rgb}     { .0 , .15, .45 }    % redefine
\definecolor{purple}{rgb}   { .6 , .1 , .75 }
\definecolor{deepblue}{rgb} { .15, .0 , .5  }
\definecolor{darkblue}{rgb} { .0 , .0 , .33 }

\usepackage{listings}    % Code-Listings

\lstdefinelanguage{PseudoCode}{
  basicstyle=\small,
  identifierstyle=\color{black},
  sensitive=false,
  comment=[l]{\#}{//}{;},
  morecomment=[s]{/*}{*/},
  commentstyle=\color{darkgrey}\ttfamily,
  stringstyle=\color{red}\ttfamily,
  morestring=[b]',
  morestring=[b]", % shorter delimiter first!
  morestring=[b]"""
}
%\usepackage{mathtools} % mainly for := sign, mathtools is not available :(
% http://tex.stackexchange.com/questions/4216/how-to-typeset-correctly
\newcommand*{\defeq}{\mathrel{\vcenter{\baselineskip0.5ex \lineskiplimit0pt
                     \hbox{\scriptsize.}\hbox{\scriptsize.}}}
                     =}
\lstset{
  % keep default basicstyle for \lstinline 
  backgroundcolor=\color{lightgray},
  columns=fixed,    % make monospace. TODO does not work?
  keepspaces=true,
  basewidth=0.60em,
  frame=shadowbox,
  rulecolor=\color{shadow},
  rulesepcolor=\color{shadow}, % shadow color
  numbers=left,
  numbersep=5pt,    % how far the line-numbers are from the code
  showspaces=false,
  showstringspaces=false,
  captionpos=b,     % bottom
  xleftmargin=2em,   % enough for 99 lines
  escapechar=!     % LaTeX inside code segments
}
\renewcommand\lstlistingname{Quelltext} % Sag Babel, wie es das Label "Listing" übersetzen soll
\renewcommand\lstlistlistingname{Quelltextverzeichnis} % Sag Babel, wie es den Titel "Listings" übersetzen soll

% short hand macro. Use like \code{var example = 1;}
\newcommand{\code}[1]{\lstinline$#1$}

% Referenzen:
\usepackage[nospace,sort]{cite}     % Zitiernummermanagement
\renewcommand{\thefootnote}{\roman{footnote}} % Römische Nummerierung für Fußnoten

% LAST IMPORT, URLs und links:
\usepackage{hyperref}    % Get clickable links, refs, figs, chapters ... SHOULD BE LAST IMPORT!
\hypersetup{
    colorlinks=false, % false doesn't work?
    linktocpage=true
    %linkcolor=blue,
    %filecolor=magenta,      
    %urlcolor=cyan,
}
% END IMPORTS



% META INFO:
\title{XXX TITEL}
\author{Arthur Jagiella \and \texttt{jaar1013@hs-karlsruhe.de}}
\date{\today}



% °_°_°_°_°_°_°_°_°_°_°_
%    BEGIN DOCUMENT:
% °_°_°_°_°_°_°_°_°_°_°_
\begin{document}

%\include{mytitle}
\title{XXX TITEL}
\maketitle



% °_°_°_°_°_°_°_°_°_°_°_
%    MAIN MATTER
% °_°_°_°_°_°_°_°_°_°_°_
\begin{abstract}
Bla bla.

Nächster Absatz. XXX
\end{abstract}
\textbf{Schlüsselworte}: Foo, Bar, 42.

\section{Einleitung} % sub-, subsub-, paragraph, subparagraph
Hallo Welt. Vgl.\cite{bestArticle}.

\subsection{Foo, quo vadis?}
Lösung:
% Code. Als externer Text-Import: \lstinputlisting[language=Python, caption={Beschreibung}, label={lst:ref}]{./pfad/zu/codeFile.py}
\begin{lstlisting}[language=PseudoCode,caption={XXX},label=lst:code1]
function code() {
  return EXIT_FAILURE;
}
\end{lstlisting}

\subsection{Bar im Wandel der Zeit}
Auf nimmer Wiedersehen, du grausame Welt

\section{XXX}
In dem nun folgenden, zeigen wir $P=NP$.

\subsection{Beweis} \label{subsec:aircontrol}
Easy

\begin{wrapfigure}[13]{Rth}{0.45\textwidth}
    \centering
    \includegraphics[width=0.45\textwidth]{./pics/pic.png}
	\caption[format=plain]{Voll der Text XXX}
	\label{fig:ref1}
\end{wrapfigure}

Wie in Bild \ref{fig:ref1} gezeigt, ... XXX

\section{Schluss}
Wie dieser Überblick dem Leser hoffentlich vermitteln konnte XXX



% °_°_°_°_°_°_°_°_°_°_°_
%    APPENDIX
% °_°_°_°_°_°_°_°_°_°_°_
\appendix

\begin{description}
\item{ABOUT}
\begin{description}
\item{Diese Arbeit} ist im Rahmen des Kurses „XXX“ an der Hochschule Karlsruhe im Wintersemester 2016/17 entstanden.

\item{Der Autor} Arthur Jagiella studiert Informatik (B.Sc.) an o.g. Hochschule.
\end{description}
\end{description}

\begin{thebibliography}{99}

\bibitem{bestArticle}
  \textsc{WIKIPEDIA}:
  \emph{Hallo-Welt-Programm}.
  https://de.wikipedia.org/wiki/Hallo-Welt-Programm.

\end{thebibliography}

\end{document}