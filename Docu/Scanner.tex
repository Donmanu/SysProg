% °_°_°_°_°_°_°_°_°_°_°_
%   PREAMBEL / IMPORTS
% °_°_°_°_°_°_°_°_°_°_°_
\documentclass[
a4paper,   %|letterpaper|a5paper|b5paper|legalpaper|executivepaper,
11pt,      %|10pt|12pt,
oneside,   %|twoside,
onecolumn, %|twocolumn,
final      %|draft,     wie Bilder eingebunden sind
]{article}

% Deutsche Sprache:
\usepackage[utf8]{inputenc} % Ermöglicht die direkte Eingabe der Umlaute.
\usepackage[T1]{fontenc}    % Wird u.a. für das Trennen von Wörtern mit Umlauten genutzt.
\usepackage[ngerman]{babel} % Wird benötigt um deutsche Bezeichnungen zu erhalten. Zum Beispiel 'Inhaltsverzeichnis'
                            % anstelle von 'Table of contents'.  Auch werden dann die Wörter gemäß der _neuen_
                            % Rechtschreibung getrennt.
\addto\captionsngerman{\renewcommand{\abstractname}{Abstract}} % Aber doch Abstract statt "Zusammenfassung"

% Bilder:
\usepackage{graphicx}    % Besserer Bilderimport
\usepackage{wrapfig}     % Erlaube Text um Bilder
\usepackage[font={small,it},labelfont={normal}]{caption}  % Bessere Bild-/Code-/Tabellenuntertitel

% Schrift und Optik:
\usepackage{geometry}
\usepackage{concrete}    % Ganz hübsche Schriftart von Donald himself
\geometry{a4paper,left=28mm,right=28mm, top=38mm, bottom=38mm}
\usepackage{setspace}
\onehalfspacing          % Anderhalbfacher Zeilenabstand
\widowpenalty = 10000
\clubpenalty = 10000

% Code:
\usepackage{color}       % Vor allem für farbigen Code
\definecolor{lightgray}{rgb}{ .95, .95, .95 }
\definecolor{shadow}{rgb}   { .75, .75, .75 }
\definecolor{darkgray}{rgb} { .4 , .4 , .4  }
\definecolor{blue}{rgb}     { .0 , .15, .45 }    % redefine
\definecolor{purple}{rgb}   { .6 , .1 , .75 }
\definecolor{deepblue}{rgb} { .15, .0 , .5  }
\definecolor{darkblue}{rgb} { .0 , .0 , .33 }

\usepackage{listings}    % Code-Listings

\lstdefinelanguage{PseudoCode}{
  basicstyle=\small,
  identifierstyle=\color{black},
  sensitive=false,
  comment=[l]{\#}{//}{;},
  morecomment=[s]{/*}{*/},
  commentstyle=\color{darkgray}\ttfamily,
  stringstyle=\color{red}\ttfamily,
  morestring=[b]',
  morestring=[b]", % shorter delimiter first!
  morestring=[b]"""
}

\lstdefinelanguage{SysProgLang}{
  basicstyle=\small,
  identifierstyle=\color{black},
  sensitive=true,
  comment=[l]{:*},
  morecomment=[s]{:*}{*:},
  commentstyle=\color{darkgray}\ttfamily,
  %stringstyle=\color{red}\ttfamily,
  %morestring=[b]',
  %morestring=[b]",
  %morestring=[b]"""
}
%\usepackage{mathtools} % mainly for := sign, mathtools is not available :(
% http://tex.stackexchange.com/questions/4216/how-to-typeset-correctly
\newcommand*{\defeq}{\mathrel{\vcenter{\baselineskip0.5ex \lineskiplimit0pt
                     \hbox{\scriptsize.}\hbox{\scriptsize.}}}
                     =}
\lstset{
  % keep default basicstyle for \lstinline 
  backgroundcolor=\color{lightgray},
  columns=fixed,    % make monospace. TODO does not work?
  keepspaces=true,
  basewidth=0.60em,
  frame=shadowbox,
  rulecolor=\color{shadow},
  rulesepcolor=\color{shadow}, % shadow color
  numbers=left,
  numbersep=5pt,    % how far the line-numbers are from the code
  showspaces=false,
  showstringspaces=false,
  captionpos=b,     % bottom
  xleftmargin=2em,   % enough for 99 lines
  escapechar=!     % LaTeX inside code segments
}
\renewcommand\lstlistingname{Quelltext} % Sag Babel, wie es das Label "Listing" übersetzen soll
\renewcommand\lstlistlistingname{Quelltextverzeichnis} % Sag Babel, wie es den Titel "Listings" übersetzen soll

% short hand macro. Use like \code{var example = 1;}
\newcommand{\code}[1]{\lstinline$#1$}

% Referenzen:
\usepackage[nospace,sort]{cite}     % Zitiernummermanagement
\renewcommand{\thefootnote}{\roman{footnote}} % Römische Nummerierung für Fußnoten

% LAST IMPORT, URLs und links:
\usepackage{hyperref}    % Get clickable links, refs, figs, chapters ... SHOULD BE LAST IMPORT!
\hypersetup{
    colorlinks=false, % false doesn't work?
    linktocpage=true
    %linkcolor=blue,
    %filecolor=magenta,      
    %urlcolor=cyan,
}
% END IMPORTS



% META INFO:
\title{Scanner}
\author{Arthur Jagiella \hspace{1cm} \texttt{jaar1013@hs-karlsruhe.de} \and Manuel Giesinger \hspace{1cm} \texttt{gima1019@hs-karlsruhe.de}}
\date{\today}



% °_°_°_°_°_°_°_°_°_°_°_
%    BEGIN DOCUMENT:
% °_°_°_°_°_°_°_°_°_°_°_
\begin{document}

%\include{mytitle}  % als externe .tex-Datei
\title{Scanner}
\maketitle



% °_°_°_°_°_°_°_°_°_°_°_
%    MAIN MATTER
% °_°_°_°_°_°_°_°_°_°_°_
\textbf{Schlüsselworte}: Scanner, Parser, Compiler, Token, Tokenizer, Buffer, Hashmap, List, Tree, Parse Tree.


\section{Einleitung} % sub-, subsub-, paragraph, subparagraph
Diese Arbeit ist im Rahmen des Kurses „Systemnahes Programmieren“ an der Hochschule Karlsruhe im Sommersemester 2017 entstanden. 

\subsection{Aufgabenstellung gesamt}
Im Kurs „Systemnahes Programmieren“ geht es inhaltlich um die Programmierung in C++ unter Verzicht auf die Nutzung von Standardbibliotheken. Die Implementierung von Puffern, verketteten Listen, Hashtabellen und Baumstrukturen soll per Hand erfolgen. Hierzu wird als Anwendung ein Compiler für eine fiktive Sprache implementiert.

\subsection{Aufgabenstellung Scanner} 
Als Teil des Compilers ist die Aufgabe des Scanners eine Source-Datei einzulesen und in ihre syntaktischen Bestandteile zu zerlegen - man pricht von \emph{tokenizing}. Dazu bedarf es zum einen eines Puffer-Speichers für den Dateiinhalt. Zum anderen ist die Token-Erkennung mithilfe eines deterministischen endlichen Automaten umgesetzt.
Parallel zur syntaktischen Analyse werden bereits erste Informationen über die Token gesammelt. Diese hält eine Symboltabelle bereit, welche als hash-map realisiert wird.


\section{Der Puffer}



\section{Der Automat}



\section{Die Symboltabelle}



\section{Der Scanner}
Die Scannerklasse fasst alle vorherigen Bestandteile zusammen und ermöglicht die Generierung einer Tokensequenz.


\subsection{Programmaufruf}
Die ausführbare ScannerTest liest als Parameter eine beliebig lange Liste von Dateinamen ein. Zum Beispiel wie in Listing \ref{lst:scantest} alle Dateien eines Ordners mit Bash-Expansion.
\begin{lstlisting}[language=PseudoCode, caption={Aufruf von ScannerTest}, label=lst:scantest]
~/$ ./proj/Scanner/debug/ScannerTest tests/*
\end{lstlisting}
Für jede Datei wird ein Scanner erzeugt, es werden alle Zeichen eingelesen und die jeweiligen Ergebnisse zusammen in eine Datei $out.txt$ geschrieben. Dies sieht dann zum Beispiel aus wie in Listing \ref{lst:scanout}.
\begin{lstlisting}[language=PseudoCode, caption={Dateiausgabe von ScannerTest}, label=lst:scanout]
TokenStop in line 1	in column 0

    --- END OF tests/empty.txt ---

TokenIdentifier in line 1	in column 1	Lexem: The
TokenIdentifier in line 1	in column 5	Lexem: Project
TokenIdentifier in line 1	in column 13	Lexem: Gutenberg
TokenIdentifier in line 1	in column 23	Lexem: EBook
\end{lstlisting}
Parallel werden mögliche Fehler und weitere Informationen auf der Console ausgegeben.

\section{Tests}
Für diverse Szenarien haben wir eine ganze Reihe von Testdateien erstellt. Einige sind im Folgenden umschrieben.

\begin{description}
\item{anything.txt} Ein potentiell sinnvolles Codebeispiel mit einer bunten Mischung verschiedener Tokens und Kommentare. Ein allgemeiner Testfall ohne besonderen Fokus. Eher als Parsertest nützlich.

\item{empty.txt} Eine komplett leere Datei, 0 Byte groß. Fokus dieses Testfalls ist die Funktion des Buffers -- also ob etwas schief geht, wenn schon zu Beginn nichts einzulesen ist.

\item{desert.txt} Eine Datei mit verschiedenen Whitespaces: Space, Newline und Tabulator. Fokus dieses Tests ist der Grenzfall, dass der Automat zwar viele Zeichen zu verarbeiten hat, aber nie ein Token entstehen sollte.

\item{EquCol2.txt} Da wir in diesem Bereich des Automaten lange Zeit Probleme hatten, die \code{line} und \code{column} korrekt anzugeben, haben wir unter \url{http://textmechanic.com/text-tools/combination-permutation-tools/permutation-generator/} alle möglichen Permutationen von $":"$, $"="$, $":="$, $"=:="$ und $":**:"$ erzeugen lassen. Ein Auszug findet sich in Quelltext \ref{lst:testEquCol2}.
\begin{lstlisting}[language=SysProgLang, caption={EquCol2.txt}, label=lst:testEquCol2]
=:=:=:=:**:
=:=::**:=:=
=:==:=::**:
=:==:=:**::
=:=:**::=:=
=:=:**:=:=:
=::=:**:=:=
=::==:=:**:
=:=:=:**::=
line10
=:=:=:=:**:
\end{lstlisting}
Anschließend haben wir (stichprobenweise) verglichen, ob
\begin{enumerate}
\item \code{line} der Anzahl der Zeilen in der Datei entspricht,
\item \code{column} niemals $< 1$ oder $> 11$ ist,
\item Die Token korrekt und an der richtigen Stelle erkannt werden.
\end{enumerate}
Prinzipiell wäre es wünschenswert, alle möglichen Kombinationen aller Tokens zu testen. Dies würde aber mit $21! = 5 \cdot 10^{19}$ Möglichkeiten unsere Kapazitäten sprengen.

\item{legalComments.txt} Dieser Test fokussiert sich auf alle erdenklichen Fälle um und in Kommentaren ($:* *:$). Innerhalb sind alle Zeichen erlaubt. Insbesondere auch $":"$ und $"*"$ allein. Desweiteren endet der letzte Kommentar nicht (bzw. mit $EOF$).

\item{theBible.txt} Dieser Test enthält den vollständigen Text der King-James-Bibel unter \url{http://www.gutenberg.org/cache/epub/10/pg10.txt} und dient vor allem für Performanz- und Speicherleck-Überprüfung.

\item{longWords.txt} Enthält Identifier mit versch. Wortlängen jenseits der doppelten Pufferbreite. Fokus ist Performanzüberprüfung der Symboltabelle.

\item{manyWords.txt} Enthält viele sich wiederholende Identifier. Fokus ist Performanz- und Konsistenzüberprüfung der Symboltabelle bei wiederholtem \code{resize()}.

\item{unknown.txt} Enthält alle dem Scanner unbekannten ASCII-Zeichen.

\item{values.txt} Enthält viele Zahlenwerte, die mit \code{strtol()} zu parsen sind.

\end{description}


%\section{Schluss}




% °_°_°_°_°_°_°_°_°_°_°_
%    APPENDIX
% °_°_°_°_°_°_°_°_°_°_°_
\appendix


\begin{description}
\item{ABOUT}
    \begin{description}
    \item{Diese Arbeit} ist im Rahmen des Kurses „Systemnahes Programmieren“ an der Hochschule Karlsruhe im Sommersemester 2017 entstanden.

    \item{Die Autoren} studieren Informatik (B.Sc.) an o.g. Hochschule.
\end{description}
\end{description}


%\begin{thebibliography}{99}
%
%\bibitem{bestArticle}
%  \textsc{WIKIPEDIA}:
%  \emph{Hallo-Welt-Programm}.
%  https://de.wikipedia.org/wiki/Hallo-Welt-Programm.
%
%\end{thebibliography}

\end{document}
